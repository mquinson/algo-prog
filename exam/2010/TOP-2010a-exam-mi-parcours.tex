%\documentclass[10pt]{article}\usepackage[correction,nu]{esial}
\documentclass[10pt]{article}\usepackage[nu]{esial}
\TOP\1A

\usepackage{amsthm,pifont,textcomp}
\usepackage{amsmath,amssymb}

\usepackage[utf8]{inputenc}
\graphicspath{{fig/}}

\begin{document}
\title{Examen intermédiaire du 28/01/2010 (1h)}
\fvset{fontsize=\footnotesize}
\maketitle

\begin{centering}
  \textbf{\large Documents interdits, à l'exception d'une feuille A4 à rendre
    avec votre copie.}

\end{centering}
\centerline{La notation tiendra compte de la présentation et de la clarté de
  la rédaction.}
\bigskip





\bigskip\QuestionCours~(5pts)

\Question(2pts) Décrivez en quelques mots  chacun des algorithmes suivants: tri à
bulle, tri par sélection, tri fusion et tri par insertion.
%
\textit{Describe each of the following algorithms: bubble sort, selection sort,
merge sort and insertion sort.} 

\Question(1pt) À quelles classes de complexité (en notation
$\Theta$) appartiennent les algorithmes 1 et 2 suivants?
%
\textit{To which complexity class (using $\Theta$ notation) do each of the
  following algorithm belong?}

\medskip\noindent\begin{minipage}{.45\linewidth}
  \begin{Verbatim}[label=algorithme 1]
pour i = 1 à n faire
  pour j = 1 à n faire
     x += 3    
  \end{Verbatim}
\end{minipage}\hfill\begin{minipage}{.45\linewidth}
  \begin{Verbatim}[label=algorithme 2]
pour i = 1 à n faire
  pour j = 1 à n faire
    x += 3
pour i = 1 à n faire
  y = x + 5
  \end{Verbatim}
\end{minipage}

\Question(1pt) Quelle est la différence entre la complexité dans le meilleure cas et
la borne inférieure de complexité? \textit{What's the difference between the
  complexity in best case and the lower bound on complexity?}

\Question(1pt) Définissez les types de récursivité suivants: terminale,
générative, mutuelle (ou croisée) et structurelle. \textit{Define the following
types of recursion: terminal, generative, mutual and structural.}



\medskip\Exercice\textbf{Code récursif mystère} (5pts). 

Considérez le code mystère suivant. \textit{Consider the following mystery code.}

\noindent\begin{minipage}{.65\linewidth}

\Question(\textonehalf pt) Explicitez les appels récursifs effectués pour 
\texttt{puzzle(25,4)}. \textit{Detail the recursive calls made for \texttt{puzzle(25,10)}}

\Question(1pt) Calculez le résultat de la fonction puzzle pour les valeurs
suivantes. \textit{Compute the result of the puzzle function for the following
  values.}
(10,1) (10,2) (10,3) (10,4) (10,6) (10,8) (10,10)\\
Que semble calculer \texttt{puzzle()}? \textit{What seems to compute
  \texttt{puzzle()}?}

\end{minipage}\hfill
\begin{minipage}{.33\linewidth}
\begin{Verbatim}[numbers=right]
public int puzzle(int i, int j) {
  if (i == 1)     
    return j;
  if (i % 2 == 1)
    return j+puzzle(i/2,j*2);
  else 
    return   puzzle(i/2,j*2);
}
\end{Verbatim}
\end{minipage}

\Question(\textonehalf pt)  Montrez la terminaison de cet algorithme. \textit{Show that
  this algorithm always terminate.}

\Question(\textonehalf pt) Quelle est la complexité algorithmique de \texttt{puzzle} (en
nombre d'appels récursifs)? \textit{What is the algorithmic complexity of
  \texttt{puzzle} (in amount of recursive calls)?}

\Question(\textonehalf pt) Est-il possible de dérécursiver directement cette
fonction ? Pourquoi ? \textit{Is it possible to transform this function
  directly into a non-recursive form? Why?}

\Question(2pts) Dérécursivez cette fonction en appliquant les méthodes vues en cours
(en une ou plusieurs étapes). Explicitez ce que vous faites et
pourquoi. \textit{Change this function into a non-recursive form (in one or
  more steps). Explicit what you are doing, and why.}

\medskip\Exercice\textbf{Encore un code mystère (mais pas récursif)} (d'après
Maylis DELEST -- 5pts).

On considère le tableau T012=\{1, 0, 2, 2, 0, 1, 0, 2, 1\} et la fonction
\texttt{swap(tab, a,b)}, qui inverse les valeurs des cases \texttt{tab[a]} et
\texttt{tab[b]}. \textit{We consider the array T012 above, and the function
  \texttt{swap(tab, a,b)} which swaps the values of the cells \texttt{tab[a]}
  and \texttt{tab[b]}}

\noindent\begin{minipage}{.69\linewidth}


\Question(1pt) Quel est le résultat de la fonction puzzle2() sur le tableau
  T012?  \textit{What is the result of this function onto T012?}
\begin{Reponse}
  T 012 = {0, 0, 0, 1, 1, 1, 2, 2, 2}  
\end{Reponse}

\Question(2pt) Dans le cas général si T est un tableau d'entiers dont les
valeurs sont des \{0, 1 ou 2\}, quel est le résultat de la fonction puzzle2()
sur T?  Argumentez votre réponse. \textit{In the general case, if T is an array
  of integers which values are in \{0,1,2\}, what is the result of this
  function onto T? Justify your answer.}

\begin{Reponse}
  La fonction trie le tableau en séparant les 0 des 1 et des 2. Après
  l'exécution  de la fonction le tableau est organisé de la
  façon suivante : une zone regroupant les 0 au début du tableau,
  suivie d'une zone de 1 et enfin une zone de 2 à  la fin du
  tableau.
\end{Reponse}

\Question(1pt) Quelle est la complexité de cette fonction ? Justifiez votre
réponse.  \textit{What is the complexity of this function? Justify.}
\begin{Reponse}
  Chaque élément du tableau n'est examiné qu'une fois et l'algorithme s'arrête
  lorsque tous les élements ont été étudiés. La complexité est donc lineaire
  par rapport au nombre d'éléments dans le tableau soit de l'ordre de n.
\end{Reponse}

\end{minipage}\hfill\begin{minipage}{.28\linewidth}
\begin{Verbatim}[gobble=2,numbers=right]
  void puzzle2(int tab[]) {
    int i=0,j=0,k=tab.length-1;
    while (i<=k) {
      if (tab[i] == 0) {
        swap(tab,i,j);
        j=j+1;
        i=i+1;
      } else if (tab[i] == 2) {
        swap(tab,i,k);
        k=k-1;
      } else {
        i=i+1;
      }
    }
  }
\end{Verbatim}
\end{minipage}


\Exercice(4pts) Écrivez les fonctions suivantes en utilisant le type
\texttt{Chaine} muni des opérations suivantes. Si votre solution ne s'exécute
pas en temps linéaire, vous serez pénalisé. \textit{Write the following
  functions using the type \texttt{Chaine} which has the following
  operations. If the time complexity of your solution is not linear, you will
  loose points.}
$$\left\{
\begin{array}{l}
  chvide:  \emptyset\mapsto Chaine\\
  estVide: Chaine\mapsto boolean\\
  premier: Chaine\mapsto Caract\grave{e}re \;\;\text{(défini ssi la chaîne 
    n'est pas vide)}\\
  reste:   Chaine\mapsto Chaine \text{\hspace{11mm}(défini ssi la chaîne 
    n'est pas vide)}\\
  adj:     Chaine\times Caract\grave{e}re \mapsto Chaine
\end{array}\right.
$$

\begin{Question}(1pt)
  $est\_membre: \left\{
    \begin{array}{l}
      Chaine\times caract\grave{e}re\mapsto bool\acute{e}en\\
      \text{retourne VRAI ssi le caractère fait partie de la chaîne}
    \end{array}\right.$  
\end{Question}

\begin{Question}(1pt)
  $somme: \left\{
    \begin{array}{l}
      Chaine\mapsto entier\\
      \text{retourne la somme de tous les éléments de la chaine (supposés
        être des chiffres)}
    \end{array}\right.$  
\end{Question}

% \begin{Question}
%   $croissante: \left\{
%     \begin{array}{l}
%       Chaine\mapsto bool\acute{e}en\\
%       \text{retourne si la chaine est croissante (dans l'ordre lexicographique)}
%     \end{array}\right.$  
% \end{Question}

\begin{Question}(2pt)
  $retourne: \left\{
    \begin{array}{l}
      Chaine\mapsto Chaine\\
      \text{retourne la chaine lue en sens inverse}
    \end{array}\right.$  
\end{Question}

\end{document}


%%% Local Variables:
%%% coding: utf-8
