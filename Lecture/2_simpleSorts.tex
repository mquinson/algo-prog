\part{Iterative Sorting Algorithms}
\label{sort}\toc
%%%%%%%%%%%%%%%%%%%%%%%%%%%%%%%%%%%%%%%%%%%%%%%%%%%%%%%%%%%%%%%%%%%%%%%%%% 
\section{Problem Specification}
%%%%%%%%%%%%%%%%%%%%%%%%%%%%%%%%%%%%%%%%%%%%%%%%%%%%%%%%%%%%%%%%%%%%%%%%%% 
\begin{frame}[squeeze]{Sorting Problem Specification}
  \begin{block}{Input data}
    \begin{itemize}
    \item A sequence of N comparable items $<a_1,a_2,a_3,\ldots,a_N>$
    \item Items are \textit{comparable} iff $\forall a, b$ in set,
      either \underline{$a<b$} or \underline{$a>b$} or \underline{$a=b$}
    \end{itemize}
  \end{block}\vspace{-.7\baselineskip}

  \begin{block}{Result}
    \begin{itemize}
    \item Permutation\footnote{reordering} $<a_1',a_2',a_3',\ldots,a_N'>$ so
      that: $a_1'\leq a_2'\leq a_3'\leq\ldots\leq a_N'$
    \end{itemize}
  \end{block}\vspace{-.7\baselineskip}

  \begin{block}{Sorting complex items}
    \begin{itemize}
    \item For example, if items represent students, they encompass name, class,
      grade
    \item \structure{Key:} value used for the sort 
    \item \structure{Extra data:} other data associated to items, permuted
      along with the keys
    \end{itemize}
  \end{block}\vspace{-.7\baselineskip}

  \begin{block}{Problem simplification}
    \begin{itemize}
    \item We assume that items are chars or integers to be sorted in
      ascending order\\ (no loss of generality)
    \end{itemize}
  \end{block}\vspace{-.7\baselineskip}

  \begin{block}{Memory consideration}
    \begin{itemize}
    \item Sort \textit{in place}, without any auxiliary array.
      Memory complexity: O(1)
    \end{itemize}
  \end{block}
\end{frame}
%%%%%%%%%%%%%%%%%%%%%%%%%%%%%%%%%%%%%%%%%%%%%%%%%%%%%%%%%%%%%%%%%%%%%%%%%% 
\section{Selection Sort}
\subsection{Presentation}
\begin{frame}{Selection Sort}
  \begin{block}{Big lines}
    \begin{itemize}
    \item First get the smallest value, and put it in first position
    \item Then get the second smallest value, and put it in second position
    \item and so on for all values
    \end{itemize}
  \end{block}

  \begin{columns}
    \begin{column}{.4\linewidth}
      \begin{block}{Example:}\bigskip
        \includegraphics<1| handout:0>[subfig=1]{fig/sort_selection.fig}
        \includegraphics<2| handout:0>[subfig=2]{fig/sort_selection.fig}
        \includegraphics<3| handout:0>[subfig=3]{fig/sort_selection.fig}
        \includegraphics<4| handout:0>[subfig=4]{fig/sort_selection.fig}
        \includegraphics<5| handout:0>[subfig=5]{fig/sort_selection.fig}
        \includegraphics<6| handout:0>[subfig=6]{fig/sort_selection.fig}
        \includegraphics<7| handout:0>[subfig=7]{fig/sort_selection.fig}
        \includegraphics<8| handout:0>[subfig=8]{fig/sort_selection.fig}
        \includegraphics<9| handout:0>[subfig=9]{fig/sort_selection.fig}
        \includegraphics<10-| handout:1>[subfig=10]{fig/sort_selection.fig}
      \end{block}
    \end{column}

    \begin{column}{.55\linewidth}
      \visible<11->{
        \begin{boitecode}{\only<11>{Pseudo}\only<12->{Scala} code}
          \structure{\visible<12->{/*} for each element, do: \visible<12->{*/}}

          \visible<12->{for (i <- 0 to length-1) \{}

          ~

          ~~\structure{\visible<12->{/*} (1) search min on [i;N] \visible<12->{*/}}

          \visible<13->{
            ~~var minpos=i
            
            ~~for (j <- i to length-1) \{ \structure{/* $\forall j\in[i;length-1]$ */}

            ~~~~if (tab(j) < tab(minpos)) \{

            ~~~~~~minpos = j

            ~~~~\}

            ~~\}
          }
          
          ~

          ~~\structure{\visible<12->{/*} (2) put min first \visible<12->{*/}}

          \visible<14->{
            ~~temp=tab(i)

            ~~tab(i)=tab(minpos)

            ~~tab(minpos)=temp

          }
          \visible<12->{\}}

          ~
        \end{boitecode}
      }
    \end{column}
  \end{columns}
\end{frame}
%%%%%%%%%%%%%%%%%%%%%%%%%%%%%%%%%%%%%%%%%%%%%%%%%%%%%%%%%%%%%%%%%%%%%%%%%% 
\subsection{Discussion}
\begin{frame}[fragile]{Selection sort discussion}
  \begin{block}{We apply a very generic approach here:}
    \begin{itemize}
    \item Do right now what you can, delay the rest for later (put min first)
    \item Progressively converge to what you are looking for (sort the remaining)
    \end{itemize}
  \end{block}

  \begin{columns}
    \begin{column}{.37\linewidth}
        \begin{boitecode}{}
for (i <- 0 to length-1) \{
  var minpos=i
  for (j <- i to length-1) \{
    if (tab(j) < tab(minpos)) \{
      minpos = j
    \}
  \}
  temp=tab(i)
  tab(i)=tab(minpos)
  tab(minpos)=temp
\}
        \end{boitecode}
    \end{column}

    \begin{column}{.6\linewidth}
      \begin{block}<2->{Memory Analysis}
        \begin{itemize}
        \item 2 extra variables\\
          {\small(only one at the same time, actually)}
        \item[$\Rightarrow$] Constant amount of extra memory
        \item[$\Rightarrow$] Space complexity is $O(1)$
        \item $O(1)$ is the smallest complexity $\leadsto$ $\Theta(1)$
        \end{itemize}
      \end{block}      
    \end{column}
  \end{columns}
  
  \begin{block}<3->{Time Analysis}
    \begin{itemize}
    \item Forget about constant times, focus on loops!
    \item<4-> Two interleaved loops which length is \textit{at most} N
    \item<5->[$\Rightarrow$] Time complexity is $O(N^2)$
    \end{itemize}
  \end{block}
\end{frame}
%%%%%%%%%%%%%%%%%%%%%%%%%%%%%%%%%%%%%%%%%%%%%%%%%%%%%%%%%%%%%%%%%%%%%%%%%% 
\begin{frame}[fragile]{Finer analysis of selection sort's time performance}
  \begin{columns}
    \begin{column}{.37\linewidth}
        \begin{boitecode}{}
for (i <- 0 to length-1) \{
  var minpos=i
  for (j <- i to length-1)
    if (tab(j) < tab(minpos))
      minpos = j
  temp=tab(i)
  tab(i)=tab(minpos)
  tab(minpos)=temp
\}
        \end{boitecode}
    \end{column}

    \begin{column}{.6\linewidth} 
      \begin{block}<+->{Best case, worst case, average case}
        \begin{itemize}
        \item No matter the order of the data,\\
          'selection sort' does the same
        \item[$\Rightarrow$] $t_{min}=t_{max}=t_{avg}$
        \end{itemize}
      \end{block}     
    \end{column}
  \end{columns}\vspace{-\baselineskip}

  \begin{block}<+->{Counting steps more precisely (but only dominant term)}
    \begin{itemize}\vspace{-.2\baselineskip}
    \item $\displaystyle T(N)=\sum_{i\in[1,N]}\left(\sum_{j\in{[i,N[}}1\right)$
      \visible<+->{$=\displaystyle\sum_{i\in[1,N]}(N-i)$
      $=\displaystyle\sum_{i\in[1,N]}N-\sum_{i\in[1,N]}i$\\\smallskip
      \hspace{9.5mm}$= N^2-\frac{N\times(N+1)}{2}$
      $= N^2-\frac{N^2+N}{2}$
      $=\frac{1}{2}N^2-\frac{1}{2}N$
      $=\frac{1}{2}(N^2-N)$}

    \item<+-> \structure{Let's prove that $T(n)\in\Omega(n^2)$.}
      For that, we want:
    \item<.-> $\exists c, n_0 / \forall N>n_0,$ 
      \framebox{$\frac{1}{2}(N^2-N)\ge cN^2$} 
      \visible<+->{
        $\!\Leftarrow\!$ \framebox{$N^2-N\ge 2cN^2$}
        $\!\Leftarrow\!$ \framebox{$N-1\ge 2cN$}
      }
    \item<+-> So, we want $\exists c, n_0 / \forall N>n_0, N\ge \frac{1}{1-2c}$
    \item<+-> Let's take anything for c ($\neq\frac{1}{2}$), and
      $n_0=\frac{1}{1-2c}$. Trivially gives what we want.
      
      % \smallskip
    \item<.->[] \centerline{\alert{$T(n)\in\Theta(n^2)$}}

    \end{itemize}
  \end{block}
\end{frame}
%%%%%%%%%%%%%%%%%%%%%%%%%%%%%%%%%%%%%%%%%%%%%%%%%%%%%%%%%%%%%%%%%%%%%%%%%%   
\section{Insertion Sort}
\subsection{Presentation}
\begin{frame}[squeeze]{Insertion Sort}
  \begin{columns}
    \begin{column}{.8\linewidth}
      \begin{block}<+->{How do you sort your card deck?}
        \begin{itemize}
        \item No human would apply \textit{selection sort} to sort a deck!
        \end{itemize}
      \end{block}\vspace{-.7\baselineskip}
      \begin{block}<.->{Algorithm used most of the time to sort a card deck:}
        \begin{itemize}
        \item[1.] If the cards \#1 and \#2 need to be swapped, do it
        \item[2.] Insert card \#3 at its position in the [1,2] part of the deck
        \item[3.] Insert card \#4 at its position in the [1,3] part of the deck
        \item[] \ldots
        \end{itemize}
      \end{block}\vspace{-.7\baselineskip}
    \end{column}

    \begin{column}{.2\linewidth}
      \includegraphics<+->[width=\linewidth]{fig/sort_insertion_cards.fig}
    \end{column}
  \end{columns}\vspace{-.4\baselineskip}


  \begin{columns}
    \begin{column}{.68\linewidth}
      \begin{block}<.->{Finding the common pattern}
        \begin{itemize}
        \item Step n ($\ge2$) is ``insert card \#(n+1) into [1,n]''
        \item<+-> Step 1 =  insert the 2. card into [1,1]
        \item<.-> We may add a Step 0 to generalize the pattern{\small\\
            (that's a no-op)}
        \end{itemize}
      \end{block}\vspace{-\baselineskip}

      \begin{alertblock}<12->{Algorithm big lines}%\medskip
      \framebox{\begin{minipage}{1.0\linewidth}\begin{tabbing}
            For each element\\
            ~~\=Find insertion position\\
            \>Move element to position
      \end{tabbing}\vspace{-1\baselineskip}\end{minipage}}
      \end{alertblock}
    \end{column}

    \begin{column}{.3\linewidth}
      \begin{block}<4->{This is \textit{Insertion Sort}}\medskip
        \includegraphics<-.| handout:0>[subfig=1]{fig/sort_insertion.fig}
        \includegraphics<+| handout:0>[subfig=2]{fig/sort_insertion.fig}
        \includegraphics<+| handout:0>[subfig=3]{fig/sort_insertion.fig}
        \includegraphics<+| handout:0>[subfig=4]{fig/sort_insertion.fig}
        \includegraphics<+| handout:0>[subfig=5]{fig/sort_insertion.fig}
        \includegraphics<+| handout:0>[subfig=6]{fig/sort_insertion.fig}
        \includegraphics<+| handout:0>[subfig=7]{fig/sort_insertion.fig}
        \includegraphics<+| handout:0>[subfig=8]{fig/sort_insertion.fig}
        \includegraphics<+| handout:0>[subfig=9]{fig/sort_insertion.fig}
        \includegraphics<+-| handout:1>[subfig=10]{fig/sort_insertion.fig}
      \end{block}      
    \end{column}
  \end{columns}
\end{frame}
%%%%%%%%%%%%%%%%%%%%%%%%%%%%%%%%%%%%%%%%%%%%%%%%%%%%%%%%%%%%%%%%%%%%%%%%%% 
\begin{frame}{Writing the \textit{insertion sort} algorithm}
  \vspace{-.5\baselineskip}
  \begin{block}{Fleshing the big lines}\smallskip
  \begin{columns}
    \begin{column}{.3\linewidth}
      \framebox{\begin{minipage}{1.0\linewidth}\small\begin{tabbing}
            For each element\\
            ~~\=Find insertion point\\
            \>Move element to position
      \end{tabbing}\vspace{-1\baselineskip}\end{minipage}}      
    \end{column}

    \begin{column}{.7\linewidth}
      \begin{itemize}\vspace{-.8\baselineskip}
      \item Finding the insertion point is easy (searching loop)
      \item Moving to position is a bit harder: ``make room''
      \item We have to \textit{shift} elements one after the other
      \end{itemize}
    \end{column}
  \end{columns}
  \end{block}

  \only<+| handout:0>{\centerline{\includegraphics[subfig=1]{fig/sort_insertion_shift.fig}}}
  \only<+| handout:0>{\centerline{\includegraphics[subfig=2]{fig/sort_insertion_shift.fig}}}
  \only<+| handout:0>{\centerline{\includegraphics[subfig=3]{fig/sort_insertion_shift.fig}}}
  \only<+| handout:0>{\centerline{\includegraphics[subfig=4]{fig/sort_insertion_shift.fig}}}
  \only<+-|
  handout:1>{\centerline{\includegraphics[subfig=5]{fig/sort_insertion_shift.fig}}}  

  \vspace{-1.5\baselineskip}
  \uncover<+->{
  \begin{itemize}
  \item Shifting elements induce a loop also
  \item We can do both searching insertion point and shifting at the same time
  \end{itemize}}

  \uncover<+->{
    \begin{columns}
      \begin{column}{.9\linewidth}
    \begin{boitecode}{}
      \visible<+->{for (i $<$- \alert{1} to length-1) \{ \structure{/* i: boundary between unsorted/sorted areas*/}}

      ~~\structure{/* save current value (it this case, that's O) */}

       \visible<+->{~~val tmp = tab(i)}

       ~~\structure{/* shift to right any element on the left being smaller than tmp */}

       \visible<+->{
         ~~var j = i

         ~~while (j$>$0 \&\& tab(j-1)$>$tmp) \{ \structure{/* while previous cell exists and is bigger */}

         ~~~~tab(j) = tab(j-1) \structure{/* copy that element */}

         ~~~~j = j -1 \structure{/* consider the next element */}

         ~~\}}

       ~~\structure{/* put tmp in cleared position */}

       \visible<+->{~~tab(j)=tmp}

       \}

    \end{boitecode}        
      \end{column}
    \end{columns}
  }
\end{frame}
%%%%%%%%%%%%%%%%%%%%%%%%%%%%%%%%%%%%%%%%%%%%%%%%%%%%%%%%%%%%%%%%%%%%%%%%%% 
\section{Bubble Sort}
\subsection{Presentation}
\begin{frame}{Bubble Sort}
  \begin{itemize}
  \item All these sort algorithms are quite difficult to write. Can we do simpler?
  \item<+-> Like ``while it's not sorted, sort it a bit''
  \end{itemize}

  \begin{columns}
    \begin{column}{.5\linewidth}
      \begin{block}<+->{Detecting that it's sorted}\medskip
        \framebox{\begin{minipage}{1.0\linewidth}\scriptsize\begin{tabbing}
          for (i $<$- 0 to length-2)\\
          ~~~\structure{/* if these two values are badly sorted */}\\
          ~~~if (tab(i)$>$tab(i+1))\\
          ~~~~~~return false\\
          return true
      \end{tabbing}\vspace{-1\baselineskip}\end{minipage}}
      \end{block}
    \end{column}

    \begin{column}{.5\linewidth}
      \begin{block}<+->{How to ``sort a bit?''}
        \begin{itemize}
        \item We may just swap these two values
        \end{itemize}
        \centerline{\framebox{\begin{minipage}{1.0\linewidth}\scriptsize\begin{tabbing}
          val tmp=tab(i)\\
          tab(i)=tab(i+1)\\
          tab(i+1)=tmp
        \end{tabbing}\vspace{-1\baselineskip}\end{minipage}}}
      \end{block}
    \end{column}
  \end{columns}
  \bigskip

  \begin{columns}
    \begin{column}{.4\linewidth}
      \begin{block}<+->{All together}
        \begin{itemize}
        \item Add boolean variable to check whether it sorted
        \end{itemize}
      \end{block}
    \end{column}
    \begin{column}{.6\linewidth}
      \visible<.->{
        \centerline{\framebox{\begin{minipage}{1.0\linewidth}\scriptsize\begin{tabbing}
          var swapped = true\\
          while (swapped) \{ \structure{/* until we do one traversal without swap */}\\
          ~~~swapped = false\\
          ~~~for (i $<$- 0 to length-2)\\
          ~~~~~~if (tab(i) $>$ tab(i+1)) \{ \structure{/* if these 2 values are badly sorted */}\\
          ~~~~~~~~~\structure{/* swap them */}\\
          ~~~~~~~~~val tmp = tab(i)\\
          ~~~~~~~~~tab(i) = tab(i+1)\\
          ~~~~~~~~~tab(i+1) = tmp\\
          ~~~~~~~~~\structure{/* and remember we swapped something */}\\
          ~~~~~~~~~swapped = true\\
          ~~~~~~\}\\
          \} \structure{/* repeat until a traversal without swapping */}
        \end{tabbing}\vspace{-1\baselineskip}\end{minipage}}}}
    \end{column}
  \end{columns}
\end{frame}
%%%%%%%%%%%%%%%%%%%%%%%%%%%%%%%%%%%%%%%%%%%%%%%%%%%%%%%%%%%%%%%%%%%%%%%%%% 
\section{Conclusion}
\begin{frame}{Conclusion on Iterative Sorting Algorithms}
  \begin{block}{Cost Theoretical Analysis}\medskip

    \begin{tabular}{|l|c|c|c|}\hline
      \structure{Amount of comparisons}&\structure{Best Case}&
      \structure{Average Case}&\structure{Worst Case}\\\hline

      Selection Sort&$O(n^2)$&$O(n^2)$&$O(n^2)$\\\hline
      Insertion Sort&$O(n)$&$O(n^2)$&$O(n^2)$\\\hline
      Bubble Sort&$O(n)$&$O(n^2)$&$O(n^2)$\\\hline 

%      \hline
%      Quick Sort&$O(n\log(n))$&$O(n\log(n))$&$O(n^2)$\\\hline
%      Merge Sort&$O(n\log(n))$&$O(n\log(n))$&$O(n\log(n))$\\\hline      
    \end{tabular}
  \end{block}

  \begin{block}{Which is the best in practice?}
    \begin{itemize}
    \item We will explore practical performance during the lab
    \item But in practice, bubble sort is \alert{awfully slow} and should never be used
    \end{itemize}    
  \end{block}

  \begin{block}<2->{Is it optimal?}
    \begin{itemize}
    \item The lower bound is $\Omega(n\log(n))$ -- cf. TD lab
    \item Some other algorithms achieve it (Quick Sort, Merge Sort)
    \item We come back on these next week
    \end{itemize}    
  \end{block}

  \begin{flushright}
    (this ends the first lecture)    
  \end{flushright}
\end{frame}