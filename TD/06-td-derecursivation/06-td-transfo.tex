\documentclass[10pt]{article}\usepackage[correction]{esial}
%\documentclass[10pt]{article}\usepackage{esial}
\TOP

\usepackage[utf8]{inputenc}
\usepackage{url}
\usepackage{amstext,amsmath,amsfonts}
\usepackage{fancyvrb}

\begin{document}
\color{black}
\title{TD6 : Dérécursivation}
\maketitle



\Exercice \textbf{Dérécursivation de fonctions sur les chaînes de caractères.}

L'objectif de cet exercice est de revenir sur les fonctions sur les chaînes
vues lors du TD2 afin de les dérécursiver. On rappelle les opérateurs de
base du type chaîne:
$$\left\{
\begin{array}{l}
  \mathtt{Nil}  \mapsto \text{La liste vide}\\
  list\mathtt{.head} \mapsto \text{Premier caractère de la liste } list
    \;\;\;\;\;\;\text{(défini ssi } list \text{ n'est pas vide)}\\
  list\mathtt{.tail} \mapsto list \text{ privée du premier élément}
    \text{\hspace{9mm}(défini ssi } list \text{ n'est pas vide)}\\
  entier \;\mathtt{::}\; list \mapsto \text{Concaténation de l'entier } entier 
    \text{ et de la liste } list
\end{array}\right.
$$

%%%%%%%% est_membre %%%%%%%%%%%%%%%%%%%%%%%%%%%%%%%%%%%%
\begin{Question}
  $est\_membre: \left\{
    \begin{array}{l}
      List\times Int \mapsto Bool\\
      \text{retourne VRAI ssi l'entier fait partie de la liste}
    \end{array}\right.$  
\end{Question}
\begin{Reponse}
  \begin{Verbatim}[label=est\_membre(ch\quotesinglbase c)]
si list = Nil alors FAUX
              sinon si list.head == i alors VRAI
                                      sinon est_membre(list.tail ,i)    
  \end{Verbatim}
  \textbf{Attention}, il ne faut pas trouver un algorithme itératif répondant à
  la question (ce qui est possible vu comment la question est dure), mais bien
  appliquer la méthode du cours sur un cas simple pour se faire la main avant
  plus compliqué.

  Ici, tout va bien, c'est récursif terminal.

  \begin{Verbatim}[label=est\_membre(ch\quotesinglbase c)]
    list_tmp = list
    while (list_tmp != Nil) {
      si ch.head = i alors return VRAI
                     sinon ch_tmp = ch.tail
    }
    return FAUX
  \end{Verbatim}
\end{Reponse}

%%%%%%%% occurence %%%%%%%%%%%%%%%%%%%%%%%%%%%%%%%%%%%%
\begin{Question}
  $occurence: \left\{
    \begin{array}{l}
      List\times Int\mapsto \mathbb{N}\\
      \text{retourne le nombre d'occurences de la valeur dans la liste}
    \end{array}\right.$  
\end{Question}
\begin{Reponse}
  \begin{Verbatim}[label=occurences(list\quotesinglbase c)]
si list = Nil alors 0
              sinon si list.head = i alors 1 + occurence(list.tail,i)
                                     sinon     occurence(list.tail,i)    
  \end{Verbatim}
  Pour dérécursiver, il faut passer sous forme terminale. Encore une fois, on
  peut aller tout droit, mais ce n'est pas l'objectif du TD. Donc, on prend le
  temps de faire une fonction avec plus d'arguments, très exactement un
  argument supplémentaire par opération réalisée lors de la remontée. Ici,
  juste une addition, donc juste un arg supplémentaire.

  Il faut également rappeller qu'on a le droit de le faire car l'addition est
  associative et commutative. On initialise l'arg supplémentaire avec l'élément
  neutre de l'opération, j'ai nommé 0.
  \begin{Verbatim}[label=occurences(list\quotesinglbase c) terminale]
occurences(list, c) = occurences_terminale(list, c, 0)

occurences_terminale(list, c, n)
  si list==Nil alors n
               sinon si list.head == c alors occurences_terminale(reste(list),c,n+1)
                                       sinon occurences_terminale(reste(list),c,n)
  \end{Verbatim}

  Si on déplie une séquence d'appel, on voit bien que les additions qui avaient
  lieu à la remontée ont maintenant lieu lors de la descente. 

  On est prets pour appliquer la recette de cuisine pour dérécursiver, qui est:
  \begin{itemize}
  \item Copie locale des arguments portant la récursion
  \item Tant que la condition du cas terminal n'est pas atteinte
    \begin{itemize}
    \item Faire les opérations réalisées dans le cas général à la descente
    \item Modifier l'argument portant la récursion
    \end{itemize}
  \item Faire le traitement du cas terminal

  \end{itemize}
  
  \begin{Verbatim}[label=occurences(list\quotesinglbase c) itérative]
  occurences(list, c) = 
    list_tmp = list // c'est lui qui porte la récursion => on copie
    n = 0 // pas besoin de copier ca, mais faut le créer pour émuler
          //  l'initialisation faite lors du lancement de helper
    while (list!=Nil ) // while (!cond_finale) 
       si premier(list) = c alors n=n+1
       list_tmp = reste(list_tmp)
    // y'a rien a faire dans le cas terminal pour occurence
  \end{Verbatim}
\end{Reponse}

%%%%%%%% retourne %%%%%%%%%%%%%%%%%%%%%%%%%%%%%%%%%%%%
\begin{Question}
  $retourne: \left\{
    \begin{array}{l}
      List\mapsto List\\
      \text{retourne la liste lue en sens inverse}
    \end{array}\right.$  
\end{Question}
\begin{Reponse}
  Là, à première vue, c'est encore plus compliquée car la version de base est
  assez difficile à dérécursiver car l'opération à la remontée est la
  concaténation ($::$), qui
  n'est pas commutative (ca veut dire A$::$B != B$::$A ;).
  \begin{Verbatim}[label=retourne(list) -- version bourinne]
si list = Nil alors Nil
            sinon dernier(list) :: retourne1(saufdernier(list))
  \end{Verbatim}

  L'idée est de changer l'algorithme, ne serait-ce que pour passer de $O(n^2)$
  à $O(n)$. \textbf{Comment leur faire trouver mieux:} Demandez leur de
  réfléchir à comment ils inversent l'ordre d'une pile de cartes : on prend une
  pile supplémentaire, on passe le premier de la pile de départ sur l'autre, et
  on recommence avec la deuxieme de la pile de départ. Si ca aide pas, faut
  détailler un exemple:

  A trier ABC $\leadsto$ \begin{tabular}{|l l|}\hline
    ABC&$\emptyset$\\
    BC&A\\
    C&BA\\
    $\emptyset$&CBA\\\hline
  \end{tabular}$\leadsto$ résulat = CBA

  \begin{Verbatim}[label=retourne(list) -- avec helper]
retourne2(list):
  retourne2_helper(list,Nil)

retourne2_helper(list_todo,list_done):
  si list_todo = Nil alors list_done
                   sinon retourne2_helper(suite(list_todo),
                                          premier(list_todo) :: list_done)
  \end{Verbatim}
  La concaténation ($::$) n'est pas commutative, mais on a remplacé les
  dernier() en premier() en meme temps qu'on changait l'ordre, alors ca va.
  L'idée est de prendre un accumulateur initialisé à ce que la récursion trouve
  au début de la remontée (la chaine vide), et faire les opérations dans l'ordre
  que la remontée l'aurait faite.  Je saurais pas vous dire ca en terme
  technique (dernier() n'est pas l'inverse de premier(), ni l'opposé, et pas
  vraiment l'opération symétrique non plus), mais ca marche bien.

  Ce qui est bô, c'est que du coup c'est récursive terminal, et on va donc
  pouvoir appliquer notre recette de cuisine.

  \begin{Verbatim}[label=retourne(list) -- itérative]
retourne_iter(list):
  list_tmp = list
  res = Nil // ce qui était l'accumulateur de la remontée
  while (list_tmp != Nil)
    premier(list_tmp) :: res
    list_tmp=suite(list_tmp)
  // tjs rien à faire au moment du cas terminal
  return res
  \end{Verbatim}

\end{Reponse}

%%%%%%%% concat %%%%%%%%%%%%%%%%%%%%%%%%%%%%%%%%%%%%
\begin{Question}
  $concat: \left\{
    \begin{array}{l}
      List\times List\mapsto List\\
      \text{le résultat est la concaténation des deux listes}
    \end{array}\right.$  
\end{Question}
\begin{Reponse}
  \begin{Verbatim}[label=version brutale: $O(n^2)$]
concat1(list1,list2):
  si list1 = Nil alors list2
               sinon concat1(saufdernier(list1),
                             dernier(list1) :: list2)
  \end{Verbatim}

  Celle-ci est rigolote puisque la version inefficace est terminale, et donc on
  peut faire une version itérative toute pourrie: 

  \begin{Verbatim}[label=version brutale: $O(n^2)$]
concat_brute_iter(list1,list2):
  list1_tmp = list1
  while (! list1_tmp = Nil)
    dernier(list1_tmp) :: list2
    list1_tmp = saufdernier(list1)
  return list2
  \end{Verbatim}

  Pour aller plus vite ($O(n^2) \rightarrow O(n)$), il faut mettre list1 à
  l'envers une bonne fois pour toute au lieu d'aller piocher le dernier à tout
  bout de champ. Exemple à donner pour qu'ils trouvent:

  \begin{tabular}{|l l l|}\hline
    ABC&DEF& en donnée\\
    CBA&DEF& on inverse list1 avant d'appeller helper\\
    BA&CDEF& récursion dans helper\\
    A&BCDEF& récursion dans helper\\
    $\emptyset$&ABCFED& Cas terminal de la récursion dans helper\\\hline
  \end{tabular}$\leadsto$ résultat = ABCDEF

  \begin{Verbatim}[label=version récursive avec helper: $O(n)$]
concat2_helper(list1,list2): 
  si list1 = Nil alors list2
               sinon concat2_helper(list1.tail,
                                    list1.head :: list2)

concat2(list1,list2):
  concat2_helper(retourne(list1),list2)    
  \end{Verbatim}

  c'est encore terminal, on peut y aller.
  \begin{Verbatim}[label=version itérative efficace: $O(n)$]
concat_iter_good(list1,list2):
  list1_tmp = retourne(list1)
  while(list1_tmp != Nil)
     list1_tmp.head :: list2
     list1_tmp = list1_tmp.tail
  // tjs rien au cas terminal
  retourne list2
  \end{Verbatim}
\end{Reponse}

%%%%%%%% difference %%%%%%%%%%%%%%%%%%%%%%%%%%%%%%%%%%%%
\begin{Question}
  $difference: \left\{
    \begin{array}{l}
      List\times List\mapsto List\\
      \text{Le résultat est la liste de tous les éléments de list1 ne faisant
        pas partie de list2}
    \end{array}\right.$    
\end{Question}
\begin{Reponse}
  Normalement, vous avez pas eu le temps de la faire, celle là, lors du TD2 vu
  qu'elle était en dernière position. C'est l'occasion ;)
  \begin{Verbatim}[label=difference(list1\quotesinglbase list2)]
si list2 = Nil alors
  list1
sinon 
  si list1 = Nil alors
    Nil
  sinon
    si est_membre(list1.head,list2) alors
      difference(list1.tail,list2)
    sinon
      list1.head :: difference(list1.tail, list2)
    finsi
  finsi
finsi    
  \end{Verbatim}

  Ce n'est pas terminal, il faut modifier. Mais si on applique la méthode de
  retourne(), ie si on introduit un accumulateur, il faut changer les premier()
  en dernier() pour faire les opérations dans le même ordre. 

  \begin{Verbatim}[label=version terminale]
difference_term(list1, list2)=
 si list2=Nil alors list1 // comme ca, plus besoin de tester à chaque coup
            sinon difference_term_help(list1,list2,Nil)

difference_term_help(list1,list2,acc):
  si list1 = Nil alors
    acc
  sinon
    si est_membre(dernier(list1),list2) alors
      difference_term_help(saufdernier(list1),list2,acc)
    sinon
      difference_term_help(saufdernier(list1),list2,dernier(list1) :: acc)
    finsi
  finsi
finsi    
  \end{Verbatim}

  On peut dérécursiver ça, mais c'est dommage de dégrader ainsi les perfs (on
  vient de passer $O(n)$ à $O(n^2)$).  On peut avoir l'idée de mettre list1 à
  l'envers avant de commencer afin d'obtenir facilement les derniers, qui se
  retrouvent devant.

  \begin{Verbatim}[label=version terminale efficace]
difference_term_good(list1, list2)=
 si list2=Nil alors list1 // comme ca, plus besoin de tester à chaque coup
            sinon difference_term_help(retourne(list1),list2,Nil)

difference_term_good_help(list1,list2,acc):
  si list1 = Nil alors
    acc
  sinon
    si est_membre(list1.head, list2) alors
      difference_term_help(list1.tail, list2,acc)
    sinon
      difference_term_help(list1.tail,list2,  list1.head :: acc)
    finsi
  finsi
finsi    
  \end{Verbatim}

  Voilà qui est mieux. On peut dérécursiver.

  \begin{Verbatim}[label=version terminale efficace]
difference_iter(list1, list2)=
 si list2=Nil alors retourne list1
 list1_tmp = retourne(list1)
 accu = Nil
 while(list1_tmp != Nil)
    si ! est_membre(list1_tmp.head, list2) alors
      list1_tmp.head :: accu
      list1_tmp = list1_tmp.tail
 // je n'arrive pas à trouver de fonction récursive faisant qqch de non trivial
 // au cas terminal.. Tant pis
 retourne accu
  \end{Verbatim}

  Une autre idée est de se dire simplement que l'énoncé n'impose pas d'ordre
  sur les lettres qu'on retourne. Peut-être bien qu'on peut alors se permettre
  de renvoyer les lettres dans le desordre (càd, utiliser list1.head sans
  prendre la peine de mettre list1 à l'envers). Dans la vraie vie, il faudrait
  vérifier, renégocier le cahier des charges avec le client, mais là, on va pas
  s'embetter.


\end{Reponse}

%%%%%%%% nnaturels %%%%%%%%%%%%%%%%%%%%%%%%%%%%%%%%%%%%
\begin{Question}
  $nnaturels: \left\{
    \begin{array}{l}
      \mathbb{N}\mapsto List\\
      \text{résultat: une liste formée des n premiers entiers naturels}
    \end{array}\right.$  
\end{Question}
\begin{Reponse}
  \fbox{\large\textbf{Question optionnelle}} a faire si vous avez décidément
  pas envie de dérécursiver Hanoi. Ce qui suit est juste la correction du
  TD2. Elle est amusante à dérécursiver, car avant, on avait du mal à faire la
  liste à l'endroit. Ici, elle va se construire tout naturellement dans
  l'accumulateur, pile comme il faut.
  
  \begin{Verbatim}[label=Version simple qui donne la liste à l'envers]
nnaturels1(n):
  si n = 0 alors Nil
           sinon n :: nnaturels1(n-1)
  \end{Verbatim}

  \begin{Verbatim}[label=Version trichée qui donne la chaine à l'endroit:]
nnaturels2(n):
  retourne(nnaturels1(list))
  \end{Verbatim}
  
  Pour faire la série dans l'ordre sans tricher, il faut une fonction
  d'aide. Pour le faire trouver, on peut écrire au tableau les différents
  arguments pris par cette fonction d'aide.
  \begin{Verbatim}[label=Version avec helper:]
nnaturels3(n):
  nnaturels3_helper(1,n-1)    

nnaturels3_helper(n, todo):
  si todo = 0 alors Nil
              sinon n :: nnaturels3_helper(n+1,todo-1)
  \end{Verbatim}
\end{Reponse}

\Exercice \textbf{Dérécursivation de l'exponentiation rapide.}

\begin{Reponse}
  Le principal intéret de cet exo est d'avoir un traitement sur le cas
  terminal. Il faudrait le retoucher (et surtout refaire la correction) pour
  expliciter ca. Peut-etre l'an prochain.
\end{Reponse}

On souhaite calculer (rapidement) $x^n$ ($x$ et $n$ étant entiers).
\begin{itemize}
\item Si $n$ est pair alors $x^n=\left(x^2\right)^{\frac{n}{2}}$. Il suffit
  alors de calculer $y^{n/2}$ avec $y=x^2$.
\item Si $n$ est impair et $n>1$, alors
  $x^n=x\times\left(x^2\right)^{\frac{n-1}{2}}$. Il suffit de calculer
  $y^{\frac{n-1}{2}}$ avec $y=x^2$ et de multiplier le résultat par $x$.
\end{itemize}

Cela nous amène à l'algorithme récursif suivant qui calcule $x^n$ pour un
entier strictement positif $n$:

$$puissance(x,n)=\left\{
\begin{array}{c l}
  x, & \text{si }n=1\\
  puissance(x^2,\frac{n}{2}), &\text{si $n$ pair}\\
  x\times puissance(x^2,\frac{n-1}{2}), &\text{si $n$ impair ($n\neq 1$)}
\end{array}\right.$$


\begin{Question}
  Écrivez une fonction récursive calculant l'exponentiel d'un entier avec cet
  algorithme.
\end{Question}
\begin{Reponse}
  \VerbatimInput{implem.java}
\end{Reponse}

\begin{Question}
  Quelle est la complexité de cet algorithme?
\end{Question}
\begin{Reponse}
  On divise le travail restant par  deux à chaque étape. La complexité est donc
  $O(log_2n)$.
\end{Reponse}

\begin{Question}
  Transformez cette fonction en une fonction récursive terminale.
\end{Question}
\begin{Reponse}
  \begin{itemize}
  \item[$\bullet$] On l'écrit sous forme fonctionnelle:
  \end{itemize}

  \begin{tabbing}
    puiss(x, n)=\\
    ~~\=\textbf{si} $n=0$ \=\textbf{alors} 1\\
    \>\>\textbf{sinon} \=\textbf{si} pair(n)\\
    \>\>\>\textbf{alors} ~~~~~puiss($x\times x$, n div 2)\\
    \>\>\>\textbf{sinon} $x\times $puiss($x\times x$, n div 2)\\
  \end{tabbing}

  \begin{itemize}
  \item[$\bullet$] Pour passer en récursivité terminale, il faut créer une
    autre fonction avec plus de paramètres. Les paramètres supplémentaires
    servent d'accumulateurs pour faire lors de la descente les calculs qui
    auraient du avoir lieu lors de la remontée.
  \item[$\bullet$] Ici, lors de la remontée, on multiplie par un nombre (pas
    toujous le même).
  \item[$\bullet$] On pose donc $puissTerm(x, n, s) = s\times puiss(x, n)$
  \item[$\bullet$] On transforme: 
    \begin{tabbing}
      $puissTerm(x, n ,s)=s\times puiss(x, n)$\\
      ~~~~~~~\=$=s\times$ (\=\textbf{si} $n=0$ \=\textbf{alors} 1\\
      \>\>\>\textbf{sinon} \=\textbf{si} pair(n)\\
      \>\>\>\>\textbf{alors} ~~~~~puiss($x\times x$, n div 2)\\
      \>\>\>\>\textbf{sinon} $x\times $puiss($x\times x$, n div 2)\\
      \>\>)
    \end{tabbing}
  \begin{itemize}
  \item[$\bullet$] On rentre le $s\times$
  \end{itemize}
    \begin{tabbing}
      ~~~~~~~=\= ~\textbf{si} $n=0$ \=\textbf{alors} $s\times 1$\\
      \>\>\textbf{sinon} \=\textbf{si} pair(n)\\
      \>\>\>\textbf{alors} ~$s\times$~~~~~puiss($x\times x$, n div 2)\\
      \>\>\>\textbf{sinon} $s\times x\times $puiss($x\times x$, n div 2)
  \end{tabbing}
  \begin{itemize}
  \item[$\bullet$] On replie (utilisant en chemin l'associativité du
    $\times$). Ca donne la définition de $puissTerm$.
  \end{itemize}
    \begin{tabbing}
      ~~~~~~~=\= ~\textbf{si} $n=0$ \=\textbf{alors} $s$\\
      \>\>\textbf{sinon} \=\textbf{si} pair(n)\\
      \>\>\>\textbf{alors} ~$puissTerm(x\times x, n\text{ div }2, s)$\\
      \>\>\>\textbf{sinon} $puissTerm(x\times x, n\text{ div }2, s\times x)$
  \end{tabbing}

  \begin{itemize}
  \item[$\bullet$] Il suffit bien entendu d'initialiser $s=1$ au démarrage de
    la récursion.
  \end{itemize}

\end{itemize}
\end{Reponse}


\begin{Question}
  Transformez la fonction obtenue en fonction itérative.
\end{Question}
\begin{Reponse}
  \VerbatimInput{expoInter.java}
\end{Reponse}

\Exercice \textbf{Dérécursivation des tours de Hanoï.}

\begin{minipage}{.4\linewidth}
\centerline{\fbox{\vbox{\large %
    \begin{tabbing}%
    \textsc{hanoi}(n,a,b):\\
      ~~\=\textbf{si} n = 1 \=\textbf{sinon} \= deplacer(a,b) \kill
      \>\textbf{si} n = 1 \>\textbf{alors} \> deplacer(a,b) \\
      \>\> \textbf{sinon} \>hanoi(n-1, a, c)\\
      \>\>                \>deplacer(a, b)\\
      \>\>                \>hanoi(n-1, c, b)\\
      \>\textbf{finsi}%
    \end{tabbing}\vspace{-.8\baselineskip}%
  }}}\medskip
  
\end{minipage}~\begin{minipage}{.55\linewidth}
  \begin{Question}
    Dérécursivez cet algorithme. Comme cet algorithme n'est pas récursif
  terminal, il faut utiliser une pile. On y conservera l'état courant du
  programme, constitué des paramètres de la fonction récursive auxquels on
  ajoute un marqueur entier indiquant le numéro de l'appel récursif simulé
  (puisqu'il y en a 2).
\end{Question}
\end{minipage}

\begin{Reponse}
  \begin{itemize}
  \item[$\bullet$] Cette dérécursivation a été vue en cours, mais c'est pas
    simple à comprendre sans écouter ;) Je vous invite à re-regarder les
    transparents 87 et 88 du cours pour voir ce qu'ils ont vu (et ce qu'il faut
    leur expliquer, donc).
  \item[$\bullet$] Pour guider les élèves pour qu'ils le trouvent, je pense
    qu'il faut tourner la version récursive à la main et leur montrer ce que
    fait l'ordinateur. Il utilise une pile pour se souvenir? Ben on va faire
    pareil. 
  \item[$\bullet$] Remarquons que cet algo est dans le cours, hein.
  \end{itemize}\begin{tabbing}
    hanoiIter(int n, int dep, int arr, int aux) \{\\
    ~~~~~\=int appel; // le numéro de l'appel récursif\\
    \>empiler(n,dep,arr,aux,1)\\
    \>while (pile non vide) \{\\
    \>~~~~~\=(n, dep, arr, aux, appel) $\leftarrow$ depiler()\\
    \>\>if ($n>0$) \{\\
    \>\>~~~~~\=if (appel == 1) \{\\
    \>\>\>~~~~~\=empiler(n, dep, arr, aux, 2)\\
    \>\>\>\>empiler(n-1, dep, aux, arr, 1)\\
    \>\>\>\} else \{ // deux valeurs possibles seulement\\
    \>\>\>\>deplace(dep, arr)\\
    \>\>\>\>empiler(n-1, aux, arr, dep, 1)\\
    \>\>\>\}\\
    \>\>\}\\
    \>\}\\
    \}
  \end{tabbing}

\end{Reponse}

\Question Dessinez les états successifs de la pile lors de Hanoi(3,a,b)

\begin{Reponse}
  Il est impossible de comprendre l'algo si on le fait pas
  tourner une fois à la main.  
\end{Reponse}

% \section{Retour sur la recherche dichotomique}
% On rappelle l'algorithme de  la recherche dichotomique dans un tableau
% trié:\medskip 

% \begin{minipage}{.5\linewidth}

% %  \centerline{\fbox{\vbox{\large %
%     \begin{verbatim}
% int dichoRec(tab,elem,inf,sup) {
%   if (sup - inf <= 1) {
%     if (tab[inf] == elem) {
%       return inf;
%     } else if (tab[sup] == elem) {
%       return sup;
%     } else {
%       return -1;
%     }
%   } else {
%     int milieu = (inf + sup) / 2;
    
%     if (elem < tab[milieu]) {
%        return dichoRec(tab,inf,milieu,elem);
%     } else {
%        return dichoRec(tab,milieu,sup,elem);
%     }
%   }
% }
% \end{verbatim}
% %}}}
% \end{minipage}~
% \begin{minipage}{.5\linewidth}
% \begin{Question}
%   Écrire la spécification pré/post.
% \end{Question}

% \begin{Question}
%   Démontrer la correction partielle de \texttt{dichoRec()} par rapport à sa
%   spécification.
% \end{Question}
% \begin{Question}
%   Démontrer que \texttt{dichoRec()} termine.
% \end{Question}

% \begin{Question}
%   Dérécursivez cette fonction.
% \end{Question}
% \end{minipage}

% \begin{Reponse}
%   \textbf{Précondition:} Soit elem n'est pas dans le tableau, soit son indice
%   est entre inf et sup. (et aussi que le tableau est trié, mais bon)

%   \textbf{Post-condition:} la même chose  
% \end{Reponse}

% \begin{Reponse}
%   \textbf{Correction partielle:} Il faut montrer que si la précondition est
%   bonne, la postcondition l'est aussi. C'est trivial avec les trois lignes de
%   la fin.
% \end{Reponse}


% \begin{Reponse}
%   \textbf{Terminaison:} On s'arrête quand $min-max<=1$. Il faut montrer que la
%   taille de la partie étudiée du tableau décroit strictement jusque là. Ce
%   n'est pas aussi trivial qu'il y parait, et il est facile de faire une boucle
%   infini en se gourant dans le calcul de milieu. Cela vient d'histoires de
%   partie entière. Mais c'est bon, là (désolé, du flou de cette réponse, c'est
%   mieux que l'an passé où y'avait pas de correction du tout pour cette
%   question, et moins bien que l'an prochain).
% \end{Reponse}


% \begin{Reponse}
%   \textbf{Dérécursivation.} Ca va tout seul, c'est à récursivité terminale,
%   faut juste l'écrire.

% \begin{verbatim}
% int dichotomie(tab, elem) {
%   sup=taille de tab
%   inf=0
%   tant que (sup-inf <=1) {
%     milieu= (inf + sup) / 2;
 
%     si (elem < tab[milieu]) {
%       sup = milieu
%     } sinon {
%       inf = milieu
%     }
%   }
%   si (tab[inf] == elem) {
%     renvoie inf
%   } sinon si (tab[sup] == elem) {
%     renvoie sup
%   } sinon {
%     renvoie -1
%   }
% }
% \end{verbatim}
% \end{Reponse}
% \url{http://www.info.univ-angers.fr/pub/stephan/LICENCE_INFO/DERECURSIVATION/derecursivation.html}

%\url{http://www-local.essi.fr/asd101/}
\end{document}
%%% Local Variables:
%%% coding: utf-8
